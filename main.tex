\documentclass{upctrans}

\usepackage{setspace}
\usepackage{amsmath}

\begin{document}

% 行距参考 https://duter2016.github.io/2020/08/09/LaTeX%E6%8E%92%E7%89%88%E5%9F%BA%E7%A1%80%E8%AF%AD%E6%B3%95/
\setstretch{1.3541667} % 任意行距

\title{第二周周报第二周周报第二周周报第二周周报第二周周报第二周周报第二周周报}
% \subtitle{这是副标题}
\author{张三}
\stuid{1707020200}
\classnum{软件工程17-02班}
\supervisor{李四}

\maketitle

\section{工作完成情况}

\subsection{推荐算法分类}

当前主流推荐算法有三类:基于内容的、协同过滤和混合推荐。

基于内容的推荐算法对内容进行分析,给用户推荐内容上类似的Item,没有冷启动问题,但存在重复推荐和很多多媒体内容难以进行分析的问题。

协同过滤算法体现了群体智慧的选择,基本思想可以说是“物以类聚,人以群分”。
其又细分为两类:基于用户的协同过滤算法(UserCF)和基于物品的协同过滤算法(ItemCF)。
基于用户的协同过滤算法有一个基本假设,即相似用户群体喜好相同;与之对应的,基于物品的协同过滤假定相似物品被同一用户喜欢。
拓展性强,算法模型精确度会随着历史数据积累而提高。但存在冷启动问题和稀疏问题。

混合推荐算法融合以上方法,通过加权、并联、串联等方式进行融合。

\subsection{协同过滤推荐算法}

\subsubsection{基于用户的协同过滤推荐算法}

UserCF以用户为研究对象,基本流程为:生成用户-物品评分矩阵、用户间相似度计算、近邻选择、推荐环节。

\paragraph{用户-物品评分矩阵}

评分矩阵是算法的基础,它主要用来收集和存储用户的偏好信息。假设有m个用户和n个物品,分别表示为$U=\{u_1,u_2,\dots,u_m\}$和$I=\{i_1,i_2,\dots,i_n\}$,用户$u$对物品$i$的评分${r_{u,i}}$,则评分矩阵

\begin{equation*}
    R=\begin{bmatrix}
        r_{1,1} & \dots r_{1,n}\\
        \vdots & \ddots & \vdots\\
        r_{m,1} & \dots & r_{m,n}
    \end{bmatrix}
\end{equation*}

\paragraph{用户间相似度计算}
\subparagraph{余弦相似度}

将用户对物品的评分看作一个$n$维向量,采用余弦公式进行相似度计算。

\begin{equation*}
    sim(u,v)
    =cos(\overline{u},\overline{v})
    =\frac{\sum\limits_{i\in{I_{u,v}}}r_{u,i}\cdot r_{v,i}}{\sqrt{\sum\limits_{i\in{I_u}}r_{u,i}^2}\sqrt{\sum\limits_{i\in{I_v}}r_{v,i}^2}}
\end{equation*}

\subparagraph{修正的余弦相似度}

利用用户评分均值作为用户评分标准,对余弦相似性方法进行修正

\begin{equation*}
    sim(u,v)=\frac{
        \sum\limits_
        {i\in{I_{u,v}}}
        (r_{u,i}-\overline{r}_u)
        (r_{v,i}-\overline{r}_v)
    }{
        \sqrt{
            \sum\limits_
            {i\in{I_u}}
            (r_{u,i}-\overline{r}_u)^2
        }
        \sqrt{
            \sum\limits_
            {i\in{I_v}}
            (r_{v,i}-\overline{r}_v)^2
        }
    }
\end{equation*}

\subparagraph{皮尔逊相关系数}

利用用户间的共同评分物品进行相似度计算


\begin{equation*}
    sim(u,v)=\frac{
        \sum\limits_
        {i\in{I_{u,v}}}
        (r_{u,i}-\overline{r}_u)
        (r_{v,i}-\overline{r}_v)
    }{
        \sqrt{
            \sum\limits_
            {i\in{I_{u,v}}}
            (r_{u,i}-\overline{r}_u)^2
        }
        \sqrt{
            \sum\limits_
            {i\in{I_{u,v}}}
            (r_{v,i}-\overline{r}_v)^2
        }
    }
\end{equation*}

\paragraph{近邻选择}

\subparagraph{KNN} 

选择用户相似度最高的k个用户作为近邻集合。

\subparagraph{阈值法}

设置阈值Q,选择相似度大于阈值Q的用户作为最近邻居。

\paragraph{推荐环节}
\subparagraph{评分预测}

综合邻居用户的评分信息进行评分预测。

\begin{equation*}
    P_{u,i}=\overline{r}_u+
    \frac{
        \sum\limits_{v\in{N(u)}}sim(u,v)\times (r_{v,i}-\overline{r}_v)
    }{
        \sum\limits_{v\in{N(u)}}sim(u,v)
    }
\end{equation*}

其中$N(u)$表示目标用户$u$的$k$近邻。

\subparagraph{推荐列表生成}

采用Top-N方法生成推荐列表即筛选出前N个评分最高的物品作为用户的推荐列表。

\subsubsection{基于物品的协同过滤推荐算法}

ItemCF以物品为研究对象,基本流程为:生成用户-物品评分矩阵、物品之间相似度计算、近邻选择、推荐环节。

\paragraph{用户-物品评分矩阵}

同UserCF算法。

\paragraph{物品相似度计算}
\subparagraph{余弦相似性}

\begin{equation*}
    sim(i,j)
    =cos(\overline{i},\overline{j})
    =\frac{\sum\limits_{u\in{U_{i,j}}}r_{u,i}\cdot r_{u,j}}{\sqrt{\sum\limits_{u\in{U_i}}r_{u,i}^2}\sqrt{\sum\limits_{u\in{U_j}}r_{u,j}^2}}
\end{equation*}

其中$U_i$和$U_j$分别表示物品$i$和$j$的已评分用户集合,$U_{i,j}$表示物品$i$和$j$的共同评分用户集合。

\subparagraph{修正的余弦相似性}

\begin{equation*}
    sim(i,j)
    =cos(\overline{i},\overline{j})
    =\frac{\sum\limits_{u\in{U_{i,j}}}
    (r_{u,i}-\overline{r}_u)(r_{u,j-\overline{r}_u)}
    }{\sqrt{\sum\limits_{u\in{U_i}}(r_{u,i}-\overline{r}_u)^2}\sqrt{\sum\limits_{u\in{U_j}}(r_{u,j}-\overline{r}_u)^2}}
\end{equation*}

\subparagraph{皮尔逊相关系数}

\begin{equation*}
    sim(i,j)=\frac{
        \sum\limits_{u\in{U_{i,j}}}
        (r_{u,i}-\overline{r}_i)
        (r_{u,j}-\overline{r}_j)
    }{
        \sqrt{
            \sum\limits_{u\in{U_{i,j}}}(r_{u,j}-\overline{r}_i)^2
        }
        \sqrt{
            \sum\limits_{u\in{U_{i,j}}}(r_{u,j}-\overline{r}_j)^2
        }
    }
\end{equation*}

\paragraph{近邻选择}

与UserCF类似。

\paragraph{推荐环节}
\subparagraph{评分预测}

\begin{equation*}
    P_{u,j}=\overline{r}_i+
    \frac{
        \sum\limits_{j\in{N(i)}}
        sim(i,j)\times (r_{u,j}-\overline{r}_j)
    }{
        \sum\limits_{v\in{N(i)}}
        sim(i,j)
    }
\end{equation*}

其中$N(i)$表示目标物品$i$的$k$近邻集合

\subparagraph{推荐列表生成}

与UserCF类似。

\subsubsection{稀疏问题}

协同过滤算法基础是评分矩阵,推荐质量和评分密度相关,但在实践中,用户规模和物品规模庞大但用户的评分数很少,严重影响算法的推荐质量。

\subsubsection{冷启动问题}

\begin{description}
    \item[用户冷启动] 对于新用户,历史评分数据很少,难以准确挖掘其兴趣偏好
    \item[物品冷启动] 新物品的评分记录很少,可能一直得不到推荐  
\end{description}

\section{下阶段工作}

编写爬虫获取小说资源。

    
\end{document}