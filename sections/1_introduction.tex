\documentclass{../upctrans}
% 公用导言区

\usepackage{setspace} % 设置间距
\usepackage{amsmath} % 数学
\usepackage{docmute} % 去除子文件导言区


\title{介绍}

\begin{document}

\section{介绍}

世界上信息的增长速度远超过我们处理能力的增长速度。我们都了解被每年都会出现的新书、杂志文章和会议记录淹没的不知所措。戏剧性地,技术既产生垃圾也产生有价值的信息,现在是时候创建一种技术来帮助我们从所有可获得的信息中筛选出对我们最有价值的。

最有希望的这种技术之一是协同过滤。协同过滤通过构建一个用户对物品的喜好数据库来工作。假设有一个新用户Neo会被匹配到数据库中来发现{\bfseries 邻居}——即和Neo历史上拥有相同喜好的用户。邻居喜欢的物品也会被推荐给Neo,因为他也可能喜欢。协同过滤算法已经在研究和实践中非常成功,包括在信息过滤应用和电商应用上。然而,现在在克服协同过滤推荐系统的两个基本挑战的过程中仍然存在一些重要的研究问题。

第一个是提高协同过滤算法的规模。这些算法现在可以实时搜索数万的潜在邻居,但现代系统的需求是在数千万的数据中搜索邻居。更进一步,现在的算法在拥有大量信息的站点的场景下有性能问题。例如,如果一个网站使用浏览模式作为内容推荐策略,它可能为活跃用户拥有数千的数据点。这些{\bfseries 长用户行}降低了每秒可以搜索的邻居的数量,更进一步导致伸缩性问题。

第二个挑战是提高为用户推荐内容的质量。用户需要可信的推荐内容来帮助他们找到推荐物品。用户会{\bfseries 用脚投票}拒绝使用总是不准确的推荐系统。

在某种程度上这两个挑战是冲突的,因为算法搜索邻居的时间越少,算法的伸缩性就越强,而推荐结果质量也就越糟糕。因此,同等看待这两个挑战非常重要,这样发现的解决方案才能既有用又实际。

在这篇文章中,我们通过应用不同的基于物品的算法解决了这些问题。传统协同过滤算法的瓶颈是在大用户量场景下搜索潜在邻居。基于物品的算法通过研究物品间关系而不是用户间关系来避免遇到这个瓶颈。为用户推荐物品是通过找到其他用户喜欢的物品的同类物品实现的。因为物品间关系可持久化,基于物品的算法可以用更少的在线计算达到和基于用户的算法的同等效果。

\subsection{相关工作}

在这个章节中我们简要呈现了有关协同过滤、推荐系统、数据挖掘和个性化的一些研究。

Tapestry是基于协同过滤的推荐系统的最早的实现之一。这个系统依赖于人们密切交流的明显观点,例如办公室工作组。然而,为大量的交流做推荐系统需要每一个人都互相了解。后来,基于排行的自动推荐系统诞生了。GroupLens研究系统为Usenet新闻和电影提供了一个伪协同过滤解决方案。Ringo and Video Recommender是一个基于电子邮件和网络的推荐音乐和电影的伟大的系统。一个特别的ACM问题呈现了大量的不同推荐系统。

其他的技术也被应用到推荐系统上,包括贝叶斯网络、聚类和Horting\footnote{译者注:不知道这个术语怎么翻译。}。

贝叶斯网络通过呈现用户信息的决策树的每一个结点和边创建一个模型。模型可以数小时或数天离线构建。模型结果非常小,非常快,寻找近邻又快速又精准。贝叶斯网络可能适合用户喜好的知识环境变化缓慢的情况但不适合用户喜好的知识环境变化迅速的环境。

聚类技术是通过识别具有相似喜好用户所在的群组实现的。一旦聚类被创建,预测单一用户可以通过平均其他用户的观点。一些聚类技术把每个用户分到不同的聚类中,然后通过分向量角度权重求聚类的均值。聚类技术通常生产的推荐物品数相对其他方法最少,而且在一些情况下,聚类技术比近邻算法拥有糟糕的精确度。一旦聚类完成,算法表现非常好,因为必须要分析的数据组数很少。在一些推荐引擎中,聚类算法也可以作为紧邻算法预处理数据集或对比邻居的第一步。因为将用户划分到聚类中可能会影响速度和用户现在所在聚类的推荐,预聚类技术可以作为准确度和实用性的重要考量。

Horting是一个基于图的技术,它的结点表示用户,而边则表示两个用户之间的相似度。通过爬取图上的临近结点并组合临近用户的观点进行预测。Horting区别于最近邻居,图上可能遍历到与用户在问题上无关的物品,因此不考虑探索临近邻居算法及物关系。在一次使用合成数据的研究中,Horting产生了比临近邻居算法更好的预测结果。

Schafer等,呈现了一个推荐系统在电商和它们如何提供一对一个性化服务的详细分类和例子。虽然这些系统在过去很成功,它们的广泛传播也暴露出了一些自身的限制,例如数据集的稀疏性问题,与高并发有关的问题等。推荐系统的稀疏性问题已经被解决。高并发问题正处于研究中,有提出降维技术的应用来解决这些问题。

我们工作的研究范围是基于物品的推荐算法——一种新的推荐算法,也用来解决这个问题。

\subsection{贡献}

这篇文章有三个主要的研究贡献:

\begin{enumerate}
    \item 分析了基于物品的算法和其子任务的不同实现。
    \item 物品相似度预计算模型来增加基于物品的推荐可伸缩性。
    \item 几个不同基于物品的算法和传统基于用户(近邻)算法的实验性比较。
\end{enumerate}

\subsection{组织}

文章剩余部分以如下方式组织。下一个章节提供了一个简要的协同过滤算法背景。我们首先正式描述了协同过滤过程然后讨论了它的两个基于内存的和基于模型的两个变体。我们接下来呈现了一些基于内存的路径的一些挑战点。在章节3中,我们呈现了基于物品的路径并详细描述了这个算法不同的子任务。章节4描述了我们的实验性工作。它提供了我们的详细数据集、评估指标、方法、不同实验的结果和关于结果的讨论。最后一个章节提供了一些总结和未来研究方向。

\end{document}
