\documentclass{upctrans}

% 公用导言区

\usepackage{setspace} % 设置间距
\usepackage{amsmath} % 数学
\usepackage{docmute} % 去除子文件导言区


\begin{document}
\begin{abstract}

    推荐系统使用知识发现技术来解决我们生活交往中遇到的个性化信息、产品和服务推荐问题。这些系统尤其是其中使用基于K临近的协同过滤算法的取得了能在网络上广为传颂的成功,而可获取信息量和网站访问用户数的激增则给推荐系统带来了严峻挑战:生产高质量的推荐内容,在一秒内完成对数百万用户和物品的推荐计算和实现数据稀疏情况下的高覆盖。传统的协同过滤推荐系统中工作量会随着参与方数量激增,因而我们需要即使在大规模场景下仍然能快速生产高质量推荐内容的新推荐系统技术。为了解决这个问题,我们需要研究基于物品的协同过滤技术。基于物品的技术首先分析用户-物品矩阵来识别不同项目之间的关系,然后使用这些关系直接为用户计算推荐物品。
        
    在这篇文章中我们分析了不同基于物品的推荐生成算法。我们调查了计算物品间相似度的不同技术(例如,物品与物品相关性还是余弦相似性)和从相似度生成推荐项的不同技术(例如,加权总和还是回归模型)。最终,我们实验性地评估了我们的结果并和基本的KNN方法做了对比。我们的实验结果表明基于物品的算法戏剧性地比基于用户的算法表现更好,即使是提供相同的时间也比最好的基于用户的算法更好。
        
    \end{abstract}
\end{document}
